%!TEX root = ./template-skripsi.tex
%-------------------------------------------------------------------------------
% 								BAB I
% 							LATAR BELAKANG
%-------------------------------------------------------------------------------

\chapter{PENDAHULUAN}

\section{Latar Belakang}
Kemajuan teknologi dewasa ini mengalami perkembangan yang sangat pesat dalam berbagai bidang. Hal tersebut ditandai dengan penerapan sistem pelayanan yang sudah terkomputerisasi di berbagai instansi, baik itu perusahaan, rumah sakit, sekolah, maupun universitas. Penerapan sistem yang terkomputerisasi pada berbagai sektor memberikan kemudahan baik bagi perusahaan yang bersangkutan maupun bagi pengguna informasi dalam rangka mencari suatu informasi dengan cepat dan tepat. Banyak perusahaan berlomba-lomba mengimplementasikan sistem yang serba terkomputerisasi yang sesuai dengan kebutuhan perusahaan tersebut dengan tujuan meningkatkan dan mengoptimalkan kualitas layanan mereka. Bahkan tidak sedikit perusahaan yang mencantumkan penerapan suatu sistem yang terkomputerisasi dalam visi misi mereka.

Salah satu sistem yang banyak digunakan di sebuah perusahaan dalam rangka mengoptimalkan kualitas layanan meraka adalah penerapan suatu sistem informasi penggajian, yang diharapkan dapat membantu dan mempermudah dalam melakukan proses penggajian karyawan. Penerapan sistem informasi penggajian tersebut sangat berpengaruh pada perusahaan atau instansi yang memiliki jumlah tenaga kerja yang cukup banyak. Lebih lanjut lagi, komponen upah atau gaji karyawan tidak hanya gaji pokok saja, tetapi juga termasuk berbagai tunjangan, insentif, potongan, lembur dan upah lain di luar gaji pokok. Besaran gaji masing-masing tenaga kerja juga berbeda. Apabila proses perhitungan gaji masih manual, dapat merugikan perusahaan dalam hal efisiensi waktu, yaitu akan memakan waktu yang cukup lama, sehingga dapat mengakibatkan keterlambatan dalam pembayaran atau pemberian gaji kepada karyawan.

Universitas Proklamasi 45 Yogyakarta adalah universitas swasta yang pengelolaannya dibawahi oleh suatu Yayasan. Dalam setiap bidang kerjanya hampir semua telah memanfaatkan sistem informasi guna mempermudah pekerjaan dan tata kelola manajerialnya. Namun dalam proses penggajian karyawan, belum efektif dan efisien karena prosesnya bersifat manual atau tidak memanfaatkan suatu sistem informasi, sehingga masih memiliki beberapa kekurangan. Kekurangan tersebut diantaranya adalah proses penggajian yang memakan waktu cukup lama, akurasi perhitungan data dan dokumentasi data tidak tertata dengan baik. Salah satu yang membuat proses penggajian memakan waktu yang cukup lama adalah kegiatan rekap data yang mempengaruhi besaran nominal gaji dihitung secara manual, seperti rekap absensi karyawan, rekap data lembur, dan rekap upah lain selain gaji pokok.  Oleh karena itu, diperlukan adanya sebuah sistem informasi penggajian karyawan yang tepat guna, efisien dan efektif untuk digunakan dalam proses perhitungan gaji karyawan di Universitas Proklamasi 45 Yogyakarta.

Dalam melakukan perancangan sistem informasi, terdapat permasalahan yang sering terjadi, yaitu permintaan perubahan \emph{requirement} dan penambahan fitur pada sistem yang diinginkan oleh klien ketika proses perancangan sistem dilakukan. Maka dari itu, diperlukan metode pengembangan sistem yang sederhana dan melibatkan klien dalam hal berkomunikasi dengan pengembang sistem. Metode \emph{Extreme Programming} merupakan salah satu metodologi \emph{Agile} yang menekankan komunikasi yang baik dengan klien, cepat dalam proses pengembangan, serta siap dalam menerima perubahan dan perbaikan setiap kali terdapat kesalahan. Oleh karena itu, metode \emph{Extreme Programming} digunakan dalam perancangan sistem informasi penggajian karyawan ini.

\section{Rumusan Masalah}
Berdasarkan latar belakang di atas, dapat dirumuskan suatu masalah sebagai berikut:
\begin{enumerate}
\itemsep0em
\item Bagaimana menganalisis dan merancang sistem informasi penggajian karyawan berbasis web di Universitas Proklamasi 45 Yogyakarta?
\item Bagaimana mengimplementasikan sistem informasi tersebut ke dalam proses penggajian karyawan di Universitas Proklamasi 45 Yogyakarta sehingga nantinya dapat menjadi sistem informasi yang tepat guna?
\item Bagaimana menganalisis efektifitas dan efisiensi dari penggunaan sistem informasi penggajian karyawan di Universitas Proklamasi 45 Yogyakarta dari segi hasil dan waktu?
\end{enumerate}

\section{Batasan Masalah}
Adapun pembatasan masalah dalam suatu penelitian sangat diperlukan agar penelitian lebih terarah dan memudahkan dalam pembahasan, sehingga tujuan penelitian dapat tercapai. Batasan masalah dalam merancang sistem informasi ini adalah sebagai berikut:
\begin{enumerate}
\itemsep0em
\item Proses perhitungan gaji pada Sistem Informasi Penggajian Karyawan ini disesuaikan dengan kebutuhan \emph{project owner};
\item Sistem Informasi Penggajian Karyawan ini memiliki beberapa kategori pengguna sesuai wewenang jabatannya dan disesuaikan dengan kebutuhan \emph{project owner}.
\end{enumerate}

\section{Tujuan Penelitian}
Berdasarkan rumusan masalah di atas, maka tujuan yang ingin diperoleh dari penelitian ini adalah:
\begin{enumerate}
\itemsep0em
\item Menganalisis dan merancang Sistem Informasi Penggajian Karyawan berbasis web di Universitas Proklamasi 45 Yogyakarta;
\item Mengimplementasikan sistem informasi tersebut ke dalam proses penggajian karyawan di Universitas Proklamasi 45 Yogyakarta sehingga menjadi sistem informasi yang tepat guna;
\item Menganalisis efektifitas dan efisiensi dari penggunaan sistem informasi penggajian karyawan di Universitas Proklamasi 45 Yogyakarta dari segi hasil dan waktu.
\end{enumerate}

\section{Manfaat Penelitian}
    Manfaat penelitian yang diharapkan yaitu:
    \begin{enumerate}
        \itemsep0em
        \item Menambah ilmu pengetahuan dan pengalaman penulis dalam mengembangkan suatu sistem informasi dari mulai tahap analisis, perancangan, implementasi dan pengujian sehingga menjadi sistem informasi yang tepat guna.
        \item Membantu dan mempermudah pihak perusahaan dalam melakukan suatu pekerjaan, contohnya proses penggajian. Kedepannya juga memungkinkan adanya pengembangan sistem lebih lanjut apabila diperlukan sesuai dengan kebutuhan.
        \item Dapat dijadikan sebagai referensi penelitian di waktu yang akan datang.
    \end{enumerate}

\section{Sistematika Penulisan}
\begin{enumerate}
    \itemsep0em
    \item \textbf{BAB I : PENDAHULUAN}
    
    Pada bab ini dijelaskan latar belakang, rumusan masalah, batasan, tujuan, manfaat, dan sistematika penulisan.
    \item \textbf{BAB II : TINJAUAN PUSTAKAN DAN LANDASAN TEORI}
    
    Pada bab ini dijelaskan teori-teori dan penelitian-penelitian sejenis sebelumnya yang digunakan sebagai acuan atau referensi dan dasar dalam melakukan penelitian.
    \item \textbf{BAB III : METODE PENGEMBANGAN SISTEM}
    
    Pada bab ini dijelaskan metode pengembangan sistem yang digunakan selama penelitian ini yang meliputi alur atau tahapan dalam melakukan perancangan sistem.
    \item \textbf{BAB IV : ANALISIS DAN PERANCANGAN}
    
    Pada bab ini dijelaskan bagaimana mengalisis objek penelitian dan permasalahan dalam penelitian serta langkah-langkah perancangan dalam menyelesaikan solusi permasalahan.
    \item \textbf{BAB V : IMPLEMENTASI DAN PENGUJIAN}
    
    Pada bab ini dijelaskan bagaimana mengimplementasikan hasil perancangan solusi permasalahan ke dalam suatu wadah yang dapat digunakan, serta menjelaskan tahapan-tahapan pengujian.
    \item \textbf{BAB VI : HASIL DAN PEMBAHASAN}
    
    Pada bab ini dijelaskan hasil dan pembahasan dari pengimplementasian penelitian dan juga hasil pengujian sistem.
    \item \textbf{BAB VII : PENUTUP}
    
    Pada bab ini dijelaskan kesimpulan dari hasil penelitian serta saran-saran yang dapat digunakan di masa yang akan mendatang ketika akan melakukan penelitian sejenis.
\end{enumerate}

% Baris ini digunakan untuk membantu dalam melakukan sitasi
% Karena diapit dengan comment, maka baris ini akan diabaikan
% oleh compiler LaTeX.
\begin{comment}
\bibliography{daftar-pustaka}
\end{comment}
