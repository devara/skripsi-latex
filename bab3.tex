%!TEX root = ./template-skripsi.tex
%-------------------------------------------------------------------------------
%                            BAB III
%               		METODOLOGI PENELITIAN
%-------------------------------------------------------------------------------

\chapter{METODE PENGEMBANGAN SISTEM}
\section{Metode Pengumpulan Data}
Dalam menyelesaikan tugas akhir ini diperlukan data dan informasi sebagai bahan yang mendukung kebenaran materi dan pembahasan. Sebelum penulisan tugas akhir ini, peneliti melakukan observasi dan wawancara terlebih dahulu untuk memperoleh data serta informasi yang dibutuhkan.
    
Penulis melakukan observasi di Universitas Proklamasi 45 Yogyakarta pada bulan Maret 2018 dengan cara meninjau dan mengamati secara langsung proses penggajian yang dilakukan disana, meliputi proses pengelolaan data pegawai, rekap absensi, rekap laporan kerja pegawai, perhitungan gaji, pembuatan slip gaji dan pembayaran gaji.
    
Wawancara dilakukan dengan cara berdiskusi serta tanya jawab dengan pihak yang dapat memberikan informasi mengenai sistem penggajian yang berjalan saat ini. Beberapa pihak yang menjadi sasaran wawancara di antaranya Ibu Hartanti Widayani selaku Kepala Bagian SDM, Mbak Shofia Kurnia Putri selaku staff bagian SDM, dan Ibu Emi Eko Sulistyowati selaku Kepala Bagian Keuangan. Wawancara tersebut dilakukan pada tanggal 2 Februari 2018 dan 3 Februari 2018 bertempat di Gedung Biro Administrasi dan Umum Universitas Proklamasi 45 Yogyakarta.
    
\section{Kebutuhan Pengembangan Sistem}
    Dalam pengembangan sistem dibutuhkan perangkat keras dan perangkat lunak. Dalam penelitian ini, perangkat keras dan perangkat lunak yang digunakan adalah sebagai berikut:
	    \begin{enumerate}
	        \itemsep0em
	        \item Perangkat Keras berupa sebuah laptop dengan spesifikasi sebeagai berikut:
	            \begin{enumerate}[label=\alph*.]
	                \itemsep0em
	                \item CPU AMD A9-9420, 3.00 GHz
	                \item RAM 4 GB
	                \item VGA Radeon R5 Integrated
	            \end{enumerate}
	        \item Perangkat Lunak diantaranya sebagai berikut:
	            \begin{enumerate}[label=\alph*.]
	                \itemsep0em
	                \item Sistem Operasi Linux distro Ubuntu 18.04
	                \item XAMPP for Linux 7.25
	                \item Sublime Text Editor 3.1.1
	                \item Visual Paradigm Community Edition 15.0
	                \item Pencil Project 3.0.4
	                \item Google Chrome 67.0.3
	                \item Codeigniter 3.1.8
	            \end{enumerate}
	    \end{enumerate}

\section{Metode Pengembangan Sistem}
\subsection{\emph{Planning} (Perencanaan)}
Pada tahapan ini peneliti mendefinisikan ruang lingkup selama penelitian, menganalisis sistem yang berjalan saat ini dan permasalahan yang terjadi. Kemudian berlanjut menganalisis kebutuhan dari sistem yang akan dibangun.

\subsection{\emph{Design} (Perancangan Desain)}
Pada tahapan ini peneliti melakukan perancangan proses kerja sistem dan perancangan basis data sesuai dengan hasil pengumpulan data dan hasil analisis yang telah dilakukan. Dalam melakukan perancangan proses kerja sistem ini, peneliti menggunakan UML karena lebih menekankan pada pengembangan sistem yang berorientasi objek.
	
\subsection{\emph{Coding} (Implementasi Kode Program)}
Pada tahapan ini merupakan tahap dimana peneliti membuat sistem berdasarkan hasil desain rancangan yang telah dibuat sebelumnya. Desain rancangan diimplementasikan ke dalam bentuk kode yang dapat dipahami oleh komputer dengan bahasa pemrograman. Seperti yang telah dijelaskan sebelumnya mengenai \emph{Extreme Programming} bahwasannya metode ini melibatkan pengguna sistem dalam proses pengembangan sistem, maka proses \emph{Coding} ini dilakukan secara berulang-ulang apabila terdapat koreksi dari pengguna sistem.
	
\subsection{\emph{Testing} (Pengujian Sistem)}
Pada tahapan ini dilakukan pengujian sistem oleh pengguna sistem untuk mengatahui apakah sistem yang dikembangkan sudah sesuai dengan fungsionalitas yang diharapkan. Pada setiap proses kerja sistem dilakukan pengujian untuk mencapai hasil akhir dari tujuan sistem diperlukan terlebih dahulu proses-proses manajemen atau input data.