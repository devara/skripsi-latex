%!TEX root = ./template-skripsi.tex
%-------------------------------------------------------------------------------
%                            	BAB VII
%               		KESIMPULAN DAN SARAN
%-------------------------------------------------------------------------------

\chapter{PENUTUP}

\section{Kesimpulan}
Penggunaan metode \emph{Extreme Programming} dalam pengembangan sistem ini, menghasilkan tiga siklus pengembangan. Dalam setiap siklusnya selalu terdapat tahapan pengujian yang melibatkan \emph{project owner} guna mengetahui apakah sistem telah sesuai dengan kebutuhan atau belum. Siklus pengembangan sistem dinyatakan selesai ketika seluruh fungsionalitas sistem yang telah direncanakan berhasil diimplementasikan dengan baik dan tidak mendapatkan koreksi dari \emph{project owner}.

Kesimpulan yang dapat diambil setelah penelitian ini berhasil dilakukan adalah sebagai berikut:
\begin{enumerate}
    \itemsep0em
    \item Penelitian ini berhasil menganalisis dan merancang sistem informasi penggajian karyawan sesuai dengan alur penggajian yang ada di Universitas Proklamasi 45 Yogyakarta berdasarkan hasil analisis yang telah dilakukan.
    \item Penelitian ini berhasil mengimplementasikan sistem informasi penggajian karyawan di Universitas Proklamasi 45 Yogyakarta sehingga menjadi sistem informasi yang tepat guna.
    \item Penelitian ini berhasil menganalisis efektifitas dan efisiensi terhadap penggunaan sistem informasi penggajian di Universitas Proklamasi 45 Yogyakarta mulai dari awal proses penginputan data-data komponen penggajian hingga didapatkan output berupa slip gaji karyawan dan laporan penggajian.
\end{enumerate}

\section{Saran}
Sistem Informasi Penggajian Karyawan ini tentunya tidak lepas dari kelemahan dan kekurangan. Oleh karena ini, peneliti menyarankan beberapa hal guna kebaikan pengembangan sistem selanjutnya, diantaranya:
\begin{enumerate}
    \itemsep0em
    \item Mengoptimalkan kinerja sistem dengan melakukan pendekatan kembali kepada pengguna sistem apabila ditemukan \emph{bug-bug} di kemudian hari.
    \item Dalam pengelolaan data absensi, sebaiknya menggunakan mesin \emph{fingerprint} yang sudah \emph{support} terhadap pengembangan sistem sehingga tidak perlu lagi adanya \emph{import} dan \emph{export} data absensi.
    \item Pengembangan lebih lanjut terhadap fitur-fitur sistem, seperti adanya fungsi pengajuan izin ataupun pengajuan cuti karyawan.
\end{enumerate}
	
	
% Baris ini digunakan untuk membantu dalam melakukan sitasi
% Karena diapit dengan comment, maka baris ini akan diabaikan
% oleh compiler LaTeX.
\begin{comment}
\bibliography{daftar-pustaka}
\end{comment}
